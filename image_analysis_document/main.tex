\documentclass{article}
\usepackage{graphicx} % Required for inserting images
\usepackage[english]{babel}

% Set page size and margins
% Replace `letterpaper' with`a4paper' for UK/EU standard size
\usepackage[letterpaper,top=2cm,bottom=2cm,left=3cm,right=3cm,marginparwidth=1.75cm]{geometry}

% Useful packages
\usepackage{amsmath}
\usepackage[colorlinks=true, allcolors=blue]{hyperref}
\usepackage{listings}

\title{FLiTrak3D}

\date{September 2025}

\begin{document}

\maketitle

\section{Introduction}


The aim of this document is to put every "trick" useful for FLiTrak3D usage, to avoid time lost when we need to reuse 4 months later :)



\url{https://git.wur.nl/cribe001/flitrak3d/-/tree/dev?ref_type=heads}





\section{Installation}


\end{document}
