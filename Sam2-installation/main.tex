\documentclass{article}

% Language setting
% Replace `english' with e.g. `spanish' to change the document language
\usepackage[english]{babel}

% Set page size and margins
% Replace `letterpaper' with`a4paper' for UK/EU standard size
\usepackage[letterpaper,top=2cm,bottom=2cm,left=3cm,right=3cm,marginparwidth=1.75cm]{geometry}

% Useful packages
\usepackage{amsmath}
\usepackage{graphicx}
\usepackage[colorlinks=true, allcolors=blue]{hyperref}
\usepackage{listings}
\title{Sam2}
\author{Camille Aracheloff}
\usepackage[most]{tcolorbox}


\begin{document}
\maketitle

%\begin{abstract}
%Your abstract.
%\end{abstract}

\section{Presentation}

Segment Anything Model 2 (SAM2)\cite{ravi2024sam} is a model developed by Meta that allows users to segment images and videos without a learning phase.\newline


\section{Installation}




%Y a t il une adapation entre espèce de canopé et de sous bois ? 

\begin{enumerate}
    \item create specific environment 
    
    \item Install drive for graphics card
    
    
    \item sam2 installation
    
    \item run !
\end{enumerate}


\subsection{Creation environment}


When we use Python for several tasks (image analysis, experiment runs), we need to use different environments to avoid conflicts between libraries. To do this: \newline


\begin{lstlisting}
    conda create --name <my-env>
\end{lstlisting}


\begin{lstlisting}
    conda create -n sam
\end{lstlisting}


To open the environment: 

\begin{lstlisting}
    conda activate sam
\end{lstlisting}



It is possible to export environment and install it from a .yaml file: 
Export when you are inside environment: 
\begin{lstlisting}
    conda export > environment.yaml
\end{lstlisting}
Or outside the environment, with myenv=sam: 

\begin{lstlisting}
    conda export --name myenv --format=environment-yaml
\end{lstlisting}



\begin{lstlisting}
    conda env create -f environment.yaml
\end{lstlisting}


\subsection{Installation with graphic card}

To use the graphic card (and improve the speed of calculation), the code needs to communicate with it; sam2 uses Pytorch. So, to use it, we need to know: 


\begin{enumerate}
    \item The compute capability of the graphic card (obtained on internet with compute capability and the model of the graphic card). \newline
    
    Based on it we can choose cuDNN, CUDA version: 
    
    \item cuDNN: \url{https://docs.nvidia.com/deeplearning/cudnn/backend/latest/reference/support-matrix.html}
    
    
    \item CUDA: \url{https://pytorch.org/get-started/locally/}
    
\end{enumerate}

SAM2 code requires python$>=3.10$, as well as torch$>=2.5.1$ and torchvision$>=0.20.1$



\begin{enumerate}
    \item install CUDA: download the version on the web site \url{https://developer.nvidia.com/cudnn}

    \item install PyTorch \url{https://pytorch.org/get-started/locally/}
    \begin{lstlisting}
        pip3 install torch torchvision --index-url https://download.pytorch.org/whl/cu126
    \end{lstlisting}

\end{enumerate}

\subsection{Installation SAM2}
\url{https://github.com/facebookresearch/sam2}




After installing the graphic card, the installation now! 


\begin{lstlisting}
    conda activate sam
    git clone https://github.com/facebookresearch/sam2.git && cd sam2
    pip install -e .
\end{lstlisting}

Go in the folder (sam2)
\begin{lstlisting}
    cd checkpoints 
    
    download_ckpts.sh
    
    
    cd ..

\end{lstlisting}
\begin{lstlisting}
    pip install spyder
\end{lstlisting}


\subsection{Running SAM2}
We use the code written by Emmanuel DENIMAL. 
The code needs to be executed in a folder without any link with sam2 folder.  

\bibliographystyle{alpha}
\bibliography{sample}

\end{document}